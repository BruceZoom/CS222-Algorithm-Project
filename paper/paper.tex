%% For double-blind review submission, w/o CCS and ACM Reference (max submission space)
% \documentclass[sigplan,10pt,review,anonymous]{acmart}\settopmatter{printfolios=true,printccs=false,printacmref=false}
\documentclass[sigplan,10pt,review]{acmart}\settopmatter{printfolios=true,printccs=false,printacmref=false}
%% For double-blind review submission, w/ CCS and ACM Reference
%\documentclass[acmsmall,review,anonymous]{acmart}\settopmatter{printfolios=true}
%% For single-blind review submission, w/o CCS and ACM Reference (max submission space)
%\documentclass[acmsmall,review]{acmart}\settopmatter{printfolios=true,printccs=false,printacmref=false}
%% For single-blind review submission, w/ CCS and ACM Reference
%\documentclass[acmsmall,review]{acmart}\settopmatter{printfolios=true}
%% For final camera-ready submission, w/ required CCS and ACM Reference
% \documentclass[acmsmall]{acmart}\settopmatter{}


%% Journal information
%% Supplied to authors by publisher for camera-ready submission;
%% use defaults for review submission.
\acmJournal{PACMPL}
\acmVolume{1}
\acmNumber{CONF} % CONF = POPL or ICFP or OOPSLA
\acmArticle{1}
\acmYear{2018}
\acmMonth{1}
\acmDOI{} % \acmDOI{10.1145/nnnnnnn.nnnnnnn}
\startPage{1}

%% Copyright information
%% Supplied to authors (based on authors' rights management selection;
%% see authors.acm.org) by publisher for camera-ready submission;
%% use 'none' for review submission.
\setcopyright{none}
%\setcopyright{acmcopyright}
%\setcopyright{acmlicensed}
%\setcopyright{rightsretained}
%\copyrightyear{2018}           %% If different from \acmYear

%% Bibliography style
\bibliographystyle{ACM-Reference-Format}
%% Citation style
%% Note: author/year citations are required for papers published as an
%% issue of PACMPL.
% \citestyle{acmauthoryear}   %% For author/year citations


%%%%%%%%%%%%%%%%%%%%%%%%%%%%%%%%%%%%%%%%%%%%%%%%%%%%%%%%%%%%%%%%%%%%%%
%% Note: Authors migrating a paper from PACMPL format to traditional
%% SIGPLAN proceedings format must update the '\documentclass' and
%% topmatter commands above; see 'acmart-sigplanproc-template.tex'.
%%%%%%%%%%%%%%%%%%%%%%%%%%%%%%%%%%%%%%%%%%%%%%%%%%%%%%%%%%%%%%%%%%%%%%


%% Some recommended packages.
\usepackage{booktabs}   %% For formal tables:
                        %% http://ctan.org/pkg/booktabs
\usepackage{subcaption} %% For complex figures with subfigures/subcaptions
                        %% http://ctan.org/pkg/subcaption
\usepackage{xcolor}
\usepackage{listings}
\lstset{
  basicstyle=\fontsize{9}{10}\selectfont\ttfamily,
  numbers=left,
  numberstyle= \tiny,
  keywordstyle= \color{ blue!70},
  commentstyle= \color{red!50!green!50!blue!50},
  frame=single,
  rulesepcolor= \color{ red!20!green!20!blue!20} ,
  escapeinside=``,
  xleftmargin=1.5em,xrightmargin=0em, aboveskip=1em,
  framexleftmargin=2em,
  showstringspaces=false,
  showtabs=false,
  breaklines=true
}
\lstdefinelanguage{Solidity}
{
  morekeywords={contract, mapping, address, uint, private, function, public, if, payable},
  morecomment=[l]{//},
  morestring=[b]"
}


\usepackage{multicol}
\usepackage{lipsum}
\usepackage{mathtools}
\usepackage{cuted}

\usepackage{amsmath}
\usepackage{extpfeil}
\usepackage{mathpartir}
\usepackage[mathscr]{eucal}

\usepackage{hyperref}
\usepackage{cleveref}
\crefformat{section}{\S#2#1#3} % see manual of cleveref, section 8.2.1
\crefformat{subsection}{\S#2#1#3}
\crefformat{subsubsection}{\S#2#1#3}

\usepackage{algorithm}
\usepackage{algorithmicx}
\usepackage{algpseudocode}
\renewcommand{\algorithmicrequire}{\textbf{Input:}}
\renewcommand{\algorithmicensure}{\textbf{Output:}}

\newcommand{\todo}[1]{\textcolor{red}{[TODO: #1]}}

\begin{document}

%% Title information
\title[Network Fusion through CDRP Model Decomposition and Assembly]{Network Fusion through CDRP Model\\ Decomposition and Assembly}         %% [Short Title] is optional;
%% when present, will be used in
%% header instead of Full Title.
%\titlenote{ }             %% \titlenote is optional;
%% can be repeated if necessary;
%% contents suppressed with 'anonymous'
%\subtitle{Subtitle}                     %% \subtitle is optional
%\subtitlenote{with subtitle note}       %% \subtitlenote is optional;
%% can be repeated if necessary;
%% contents suppressed with 'anonymous'


%% Author information
%% Contents and number of authors suppressed with 'anonymous'.
%% Each author should be introduced by \author, followed by
%% \authornote (optional), \orcid (optional), \affiliation, and
%% \email.
%% An author may have multiple affiliations and/or emails; repeat the
%% appropriate command.
%% Many elements are not rendered, but should be provided for metadata
%% extraction tools.

\author{Zhongye Wang}
% \authornote{Supervised by Qinxiang Cao, Shanghai Jiao Tong University, John Hopcroft Center for Computer Science.}          %% \authornote is optional;
%% can be repeated if necessary
%\orcid{nnnn-nnnn-nnnn-nnnn}             %% \orcid is optional
\affiliation{
	%\position{Position2b}
	%\department{Department2b}             %% \department is recommended
	\institution{Shanghai Jiao Tong University}           %% \institution is required
	%\streetaddress{Street3b Address2b}
	%\city{City2b}
	%\state{State2b}
	%\postcode{Post-Code2b}
	%\country{Country2b}                   %% \country is recommended
}
% \email{wangzhongye1110@sjtu.edu.cn}          %% \email is recommended

\author{Yichen Xie}
% \authornote{Supervised by Qinxiang Cao, Shanghai Jiao Tong University, John Hopcroft Center for Computer Science.}          %% \authornote is optional;
%% can be repeated if necessary
%\orcid{nnnn-nnnn-nnnn-nnnn}             %% \orcid is optional
\affiliation{
	%\position{Position2b}
	%\department{Department2b}             %% \department is recommended
	\institution{Shanghai Jiao Tong University}           %% \institution is required
	%\streetaddress{Street3b Address2b}
	%\city{City2b}
	%\state{State2b}
	%\postcode{Post-Code2b}
	%\country{Country2b}                   %% \country is recommended
}
% \email{}          %% \email is recommended

\author{Xinyu Zhan}
% \authornote{Supervised by Qinxiang Cao, Shanghai Jiao Tong University, John Hopcroft Center for Computer Science.}          %% \authornote is optional;
%% can be repeated if necessary
%\orcid{nnnn-nnnn-nnnn-nnnn}             %% \orcid is optional
\affiliation{
	%\position{Position2b}
	%\department{Department2b}             %% \department is recommended
	\institution{Shanghai Jiao Tong University}           %% \institution is required
	%\streetaddress{Street3b Address2b}
	%\city{City2b}
	%\state{State2b}
	%\postcode{Post-Code2b}
	%\country{Country2b}                   %% \country is recommended
}
% \email{}          %% \email is recommended



%% Abstract
%% Note: \begin{abstract}...\end{abstract} environment must come
%% before \maketitle command
\begin{abstract}
  The abstract \dots
\end{abstract}


%% 2012 ACM Computing Classification System (CSS) concepts
%% Generate at 'http://dl.acm.org/ccs/ccs.cfm'.

%% End of generated code


%% Keywords
%% comma separated list
\keywords{}  %% \keywords are mandatory in final camera-ready submission


%% \maketitle
%% Note: \maketitle command must come after title commands, author
%% commands, abstract environment, Computing Classification System
%% environment and commands, and keywords command.
\maketitle

\section{Introduction}
The introduction \dots

\subsection{Our Work}

\section{CDRP Model Decomposition and Clustering}
\subsection{Critical Data Routing Path Model}
\subsection{CDRP Model Clustering}

\section{CDRP Model Assembly}
\subsection{The Assembly Problem}
For a clear formulation, we define the input of a problem instance to have the following components:
\begin{itemize}
	\item $L$ class labels, denoted $\{l_j\}$
	\item $N$ subclasses in total \begin{itemize}
		\item $M$ target subclasses, denoted $\mathcal{A} = \{A_i\}$
		\item $N-M$ dummy subclasses (with irrelevant class labels), denoted $\bar{\mathcal{A}} = \{A_i\}$
	\end{itemize}
	\item $g(A_i) := l_j$ the partial map from $A_i$ to class label $l_j$
	\item $K$ classifiers, denoted $\mathcal{D} = \{d_j\}$
	\item $f(A_i, d_j) := 0, 1$ the classifier certificate that $A_i$ is classified positive in classifier $d_j$
	\item $score(d_j)$ the accuracy(?) of classifier $d_j$
	\item $k$ the maximum number of classifier to use
\end{itemize}

The solution to the instance should contain:
\begin{itemize}
	\item $D \subset \mathcal{D}$ a set of $m$ classifiers \textit{covering} all $M$ target subclasses s.t. $m < k$
	\item $score(D)$ the overall accuracy(?) of the classifier set
\end{itemize}

We define the concept of \textit{covering} as follows.
\begin{definition}
	A set of classifiers covers $M$ target subclasses iff. given any sample, the joint classifier can determine whether it \textit{solely} belongs to \begin{enumerate}
		\item a subset (with consistent labels) of the $M$ target subclasses, or,
		\item the rest dummy subclasses $\bar{\mathcal{A}}$.
	\end{enumerate}
\end{definition}

We start with analyzing the intractability of the problem.
\begin{theorem}
	The assembly problem is an NP problem.
\end{theorem}
\begin{proof}
We prove the theorem by giving a polynomial algorithm for the certificate problem.

Given a certificate $D \subset \mathcal{D}$, we encode each subclass with a binary string $s$ of length $k$, where the $j$-th element for the $i$ subclass is $f(A_i, d_j)$ ($d_j \in D$).
This step takes $O(Nk) = O(NK)$ time.
The algorithm returns $yes$ iff. \begin{enumerate}
	\item No two subclasses in $\mathcal{A}$ share the same encoding but have different labels, and,
	\item No subclass in $\mathcal{A}$ shares the same encoding with any subclass in $\bar{\mathcal{A}}$.
\end{enumerate}
Otherwise, it returns $no$.
This step takes $O(N)$ if we use hash table to record the occurrence of encoded strings.

% \begin{algorithm}[h]
% 	\caption{}
% 	\begin{algorithmic}
% 		\Require A assembly problem instance $k$; a certificate $D \subset \mathcal{D}$.
% 		\Ensure Whether problem instance $k$ is solvable.
% 		\State initialize a hash table $H$ with default value -1;
% 		\For{$i = 1, \cdots, N$}
% 			\For{$d_j \in D$}
% 				\State $s_{i,j} \leftarrow f(A_i, d_j)$;
% 			\EndFor
% 			\If{$H[s_i]$ != $g(A_i)$}
% 				\State \Return $no$
% 			\ElsIf{$H[s_i]$ == -1}
% 				\State a
% 			\EndIf
% 		\EndFor
% 	\end{algorithmic}
% \end{algorithm}

We then show the algorithm returns $yes$ iff. $D$ is a certificate to the problem.
\begin{enumerate}
	\item[$\Rightarrow$:]
	The encoding for each target subclass is different from that of another subclass having different labels, meaning the classifier set can distinguish them from subclasses with other labels, and it is then clear that $D$ covers all target subclasses.
	\item[$\Leftarrow$:]
	Assume the algorithm returns $no$, then
	\begin{enumerate}
		\item Two subclasses with different labels share the same encoding, i.e., the classifier set cannot determine which label to assign for both subclasses, which violates the first condition of the "covering";
		\item Some target subclass (with meaningful label) share the same encoding with some dummy subclasses.
		The classifier set cannot determine whether samples of the target subclass belong to the dummy class or not, which violates the second condition.
	\end{enumerate}
	Therefor, $D$ is not a certificate to the problem.
\end{enumerate}
As a result, the problem instance $k$ is solvable iff. there exists a $D$ with size $k$ such that the algorithm returns $yes$.
By definition, the problem is an NP problem.
\end{proof}

\subsection{The Algorithm}

\section{Experiments}
\subsection{CDRP Models of VGGNet and ? trained on CIFAR-100}
\subsection{Fusing VGGNet and ? for Specific Tasks through CDRP Assembly}

\section{Related Work}
The related work \dots

\section{Conclusion}
The conclusion \dots

% Acknowledgments

% Bibliography
% \bibliographystyle{acm}
% \bibliographystyle{unsrt}
\bibliography{../full_list.bib}


%% Appendix
%\appendix
%\section{Appendix}

%Text of appendix \ldots

\end{document}