%% For double-blind review submission, w/o CCS and ACM Reference (max submission space)
% \documentclass[sigplan,10pt,review,anonymous]{acmart}\settopmatter{printfolios=true,printccs=false,printacmref=false}
%% For double-blind review submission, w/ CCS and ACM Reference
%\documentclass[acmsmall,review,anonymous]{acmart}\settopmatter{printfolios=true}
%% For single-blind review submission, w/o CCS and ACM Reference (max submission space)
%\documentclass[acmsmall,review]{acmart}\settopmatter{printfolios=true,printccs=false,printacmref=false}
%% For single-blind review submission, w/ CCS and ACM Reference
%\documentclass[acmsmall,review]{acmart}\settopmatter{printfolios=true}
%% For final camera-ready submission, w/ required CCS and ACM Reference
\documentclass[acmsmall,nonacm]{acmart}\settopmatter{}


%% Journal information
%% Supplied to authors by publisher for camera-ready submission;
%% use defaults for review submission.
\acmJournal{PACMPL}
\acmVolume{1}
\acmNumber{CONF} % CONF = POPL or ICFP or OOPSLA
\acmArticle{1}
\acmYear{2018}
\acmMonth{1}
\acmDOI{} % \acmDOI{10.1145/nnnnnnn.nnnnnnn}
\startPage{1}

%% Copyright information
%% Supplied to authors (based on authors' rights management selection;
%% see authors.acm.org) by publisher for camera-ready submission;
%% use 'none' for review submission.
\setcopyright{none}
%\setcopyright{acmcopyright}
%\setcopyright{acmlicensed}
%\setcopyright{rightsretained}
%\copyrightyear{2018}           %% If different from \acmYear

%% Bibliography style
\bibliographystyle{ACM-Reference-Format}
%% Citation style
%% Note: author/year citations are required for papers published as an
%% issue of PACMPL.
% \citestyle{acmauthoryear}   %% For author/year citations


%%%%%%%%%%%%%%%%%%%%%%%%%%%%%%%%%%%%%%%%%%%%%%%%%%%%%%%%%%%%%%%%%%%%%%
%% Note: Authors migrating a paper from PACMPL format to traditional
%% SIGPLAN proceedings format must update the '\documentclass' and
%% topmatter commands above; see 'acmart-sigplanproc-template.tex'.
%%%%%%%%%%%%%%%%%%%%%%%%%%%%%%%%%%%%%%%%%%%%%%%%%%%%%%%%%%%%%%%%%%%%%%


%% Some recommended packages.
\usepackage{booktabs}   %% For formal tables:
                        %% http://ctan.org/pkg/booktabs
\usepackage{subcaption} %% For complex figures with subfigures/subcaptions
                        %% http://ctan.org/pkg/subcaption
\usepackage{xcolor}
\usepackage{listings}
\lstset{
  basicstyle=\fontsize{8}{10}\selectfont\ttfamily,
  numbers=left,
  numberstyle= \tiny,
  keywordstyle= \color{ blue!70},
  commentstyle= \color{red!50!green!50!blue!50},
  frame=single,
  rulesepcolor= \color{ red!20!green!20!blue!20} ,
  escapeinside=``,
  xleftmargin=1.5em,xrightmargin=0em, aboveskip=1em,
  framexleftmargin=2em,
  showstringspaces=false,
  showtabs=false,
  breaklines=true
}
\lstdefinelanguage{Solidity}
{
  morekeywords={contract, mapping, address, uint, private, function, public, if, payable},
  morecomment=[l]{//},
  morestring=[b]"
}

% \usepackage{biblatex}

\usepackage{multicol}
\usepackage{lipsum}
\usepackage{mathtools}
\usepackage{cuted}

\usepackage{amsmath}
\usepackage{extpfeil}
\usepackage{mathpartir}
\usepackage[mathscr]{eucal}

\usepackage{caption}

\usepackage{hyperref}
\usepackage{cleveref}

% \usepackage[style=authoryear-comp]{biblatex}
% \DeclareLabeldate[online]{%
%   \field{date}
%   \field{year}
%   \field{eventdate}
%   \field{origdate}
%   \field{urldate}
% }
% \DeclareLabeldate{\field{date}\field{eventdate} \field{origdate}\literal{nodate}}

% \DefineBibliographyStrings{english}{%
% nodate = {\ifboolexpr{test{\ifentrytype{misc1}} or test{\ifentrytype{misc5}}}{}{o\adddot D\adddot}},
% }

\crefformat{section}{\S#2#1#3} % see manual of cleveref, section 8.2.1
\crefformat{subsection}{\S#2#1#3}
\crefformat{subsubsection}{\S#2#1#3}


\begin{document}

%% Title information
\title[]{CS222 Project Proposal}         %% [Short Title] is optional;
%% when present, will be used in
%% header instead of Full Title.

\author{Yichen Xie}
% \authornote{Supervised by Qinxiang Cao, Shanghai Jiao Tong University, John Hopcroft Center for Computer Science.}          %% \authornote is optional;
%% can be repeated if necessary
%\orcid{nnnn-nnnn-nnnn-nnnn}             %% \orcid is optional
\affiliation{
	%\position{Position2b}
	%\department{Department2b}             %% \department is recommended
	\institution{Shanghai Jiao Tong University}           %% \institution is required
	%\streetaddress{Street3b Address2b}
	%\city{City2b}
	%\state{State2b}
	%\postcode{Post-Code2b}
	%\country{Country2b}                   %% \country is recommended
}
% \email{}          %% \email is recommended

\author{Zhongye Wang}
% \authornote{Supervised by Qinxiang Cao, Shanghai Jiao Tong University, John Hopcroft Center for Computer Science.}          %% \authornote is optional;
%% can be repeated if necessary
%\orcid{nnnn-nnnn-nnnn-nnnn}             %% \orcid is optional
\affiliation{
	%\position{Position2b}
	%\department{Department2b}             %% \department is recommended
	\institution{Shanghai Jiao Tong University}           %% \institution is required
	%\streetaddress{Street3b Address2b}
	%\city{City2b}
	%\state{State2b}
	%\postcode{Post-Code2b}
	%\country{Country2b}                   %% \country is recommended
}
\email{wangzhongye1110@sjtu.edu.cn}          %% \email is recommended

\author{Xinyu Zhan}
% \authornote{Supervised by Qinxiang Cao, Shanghai Jiao Tong University, John Hopcroft Center for Computer Science.}          %% \authornote is optional;
%% can be repeated if necessary
%\orcid{nnnn-nnnn-nnnn-nnnn}             %% \orcid is optional
\affiliation{
	%\position{Position2b}
	%\department{Department2b}             %% \department is recommended
	\institution{Shanghai Jiao Tong University}           %% \institution is required
	%\streetaddress{Street3b Address2b}
	%\city{City2b}
	%\state{State2b}
	%\postcode{Post-Code2b}
	%\country{Country2b}                   %% \country is recommended
}
% \email{}          %% \email is recommended


% %% Abstract
% %% Note: \begin{abstract}...\end{abstract} environment must come
% %% before \maketitle command
% \begin{abstract}
% 	Text of abstract \ldots
% \end{abstract}


%% 2012 ACM Computing Classification System (CSS) concepts
%% Generate at 'http://dl.acm.org/ccs/ccs.cfm'.

%% End of generated code


%% Keywords
%% comma separated list
\keywords{}  %% \keywords are mandatory in final camera-ready submission


%% \maketitle
%% Note: \maketitle command must come after title commands, author
%% commands, abstract environment, Computing Classification System
%% environment and commands, and keywords command.
\maketitle

% xyc, until the last
\section{Introduction}

% xyc
\section{Problem Description}

% wzy
\section{Problem Formulation}
There has not been much time for us to fully consider the optimal way of addressing those problems, but we do propose one possible formulation and the solution idea in this section.

Assume there are $N$ classes in total, each denoted by $A_i$.
We further divide each class $A_i$ into $M_i$ subsets $A_{ij}$ according to some features of samples, e.g., the activation of certain convolution filters.
Now, we have a set $\mathcal{D}$ of $K$ (a relatively large number) binary classifiers each identifying a subset of all $A_{ij}$ classes.
Say classifier $d$ separates $\{A_{ij} \,|\, (i, j) \in I_d\}$ from the rest of classes, i.e., given any sample, it determines wether the label of the sample belongs to $I_d$.
Here, $I_d$ is the positive label set defined by the classifier $d$.
Note that we can always define another classifier whose positive label set is exactly the negative label set of the classifier $d$.

Our task is to find a subset $D^* \subset \mathcal{D}$, such that for any class $A_i$, there exists a partition that divides it into some subclasses $S_i^{(l)}$ where $A_i = \bigcup_l \{A_{ij} \,|\, (i, j) \in S_i^{(l)}\}$, and for each subclass $S_i^{(l)}$ there exists a subset of classifiers $D^{(l)} \subseteq D^*$ where $S_i^{(l)} = \bigcap_d \{I_d \,|\, d \in D^{(l)}\}$.
Here, subclass $S_i^{(l)}$ represents the union of some subsets $A_{ij}$ of class $A_i$, and the union of all subclasses of $A_i$ should reconstruct the $A_i$ itself.
The optimal set of classifiers $D^*$ should allow us to determine wether a sample belongs to any given subclass $S_i^{(l)}$ using a subset $D^{(l)}$ in it, which in turn determines wether the sample belongs to $A_i$.
Figure~\ref{fig:feasible-example} shows a feasible classifier set $D$ in a simple example.

\begin{figure}[h]
    \centering
    \includegraphics[width=0.45\linewidth]{fig/formulation_example.png}
    \caption{An Example of Feasible Classifier Set:
    The samples in class $A_1$ is further divided into $A_{11}, A_{12}$ and those in class $A_2$ is further divided into $A_{21}, A_{22}, A_{23}$.
    We do not divide $A_3$ but rename it to $A_{31}$.
    There are 4 discriminators $d_1, d_2, d_3, d_4$ respectively identify $\{A_{11}, A_{12}, A_{21}, A_{22}\}$, $\{A_{21}, A_{31}\}$, $\{A_{22}, A_{23}\}$, and $\{A_{23}, A_{31}\}$.
    We have: \textit{(a)} the intersection of $d_1$ and the compliments of $d_2, d_3$ identifies $A_1 = A_{11} \cup A_{12}$;
    \textit{(b)} the intersection of $d_1$ and $d_2$ identifies $A_{21}$, and its union with $d_3$ determines the entire $A_2$;
    \textit{(c)} the intersection of $d_3$ and the compliment of $d_1$ determines $A_3 = A_{31}$.
    Clearly, $D = \{d_1, d_2, d_3\}$ is a feasible set, where the redundant classifier $d_4$ is not selected.}
    \label{fig:feasible-example}
\end{figure}

There are further constraints and considerations of the optimal set of classifiers $D^*$.
\begin{itemize}
    \item \textbf{Number of Classifiers.} We need to choose adequate number of classifiers $|D^*|$ so that it covers all classes. But we prefer to use as less classifiers as possible to save computational consumptions.
    \item \textbf{Network Size.} The size for each classifier is limited. Large networks as classifiers improve accuracy but slow down inferences. Therefore, we will have penalty for large classifiers.
    \item \textbf{Overall Accuracy.} We know the accuracy of each classifier as a prior, but we need to derive the overall accuracy based on it. The $D^*$ should optimize the overall accuracy along with other goals.
\end{itemize}
These are factors we will consider in the later development and we are open for other suggestions.

% zxy
\section{Related Work}


%% Acknowledgments

% Bibliography
% \bibliographystyle{acm}
% \bibliographystyle{unsrt}
\bibliography{../full_list.bib}


%% Appendix
%\appendix
%\section{Appendix}

%Text of appendix \ldots

\end{document}